\documentclass[openany]{scrartcl}
%\usepackage{draftwatermark}  %%%%Marca de agua
%\setWatermarkText{
%Marca de Agua
%}
%\SetWatermarkColor{gray!30}
%\SetWatermarkScale{1.5}

%\usepackage{profcollege}

\usepackage{xparse}

\usepackage{dingbat} %\smallpencil
\usepackage{xlop}

\usepackage[spanish,es-noshorthands]{babel}
\usepackage[utf8]{inputenc}
\usepackage[T1]{fontenc}

\usepackage{expl3}

\ExplSyntaxOn
\renewcommand\thepage{\int_to_bin:n{\value{page}}}  %En binario
\ExplSyntaxOff  

\usepackage{tcolorbox} % Cajas
\usepackage{amsmath}  % Simbolos matematicos


\usepackage{xcolor}                %Colores adicionales


\usepackage[papersize={216mm,330mm},tmargin=5mm,bmargin=5mm,lmargin=5mm,rmargin=5mm]{geometry}

\usepackage{clock}
\usepackage[clock]{ifsym}
\usepackage{amsmath}
\usepackage{amssymb}

\usepackage{multicol}
\usepackage{hyperref}
\hypersetup{
    colorlinks=true,
    linkcolor=blue,
    filecolor=magenta,      
    urlcolor=red,
}

%%%%%%%%%%%%%%%%%%%%%%Numeracion de pagina
\usepackage{scrlayer-scrpage} 
\usepackage{tikzducks}


%\setlength{\footheight}{100pt}
%\cfoot{%
%    \shuffleducks
%    \begin{tikzpicture}[scale=0.8] 
%        \duck[signpost=\scalebox{0.6}{\thepage},\randomhead]
%    \end{tikzpicture}
%} 
%\pagestyle{scrheadings}
%%%%%%%%%%%%%%%%%%%%%%%%%%

\renewcommand{\thesection}{\Alph{section}} %Contador Alfabetico para las secciones

\usepackage{titlesec}
\titleformat{\section}[frame]
{\normalfont\bfseries\Large}
{\filright
%\footnotesize
\enspace \thesection \enspace}
{8pt}
{\Large\bfseries\filcenter}

\titleformat{\subsection}[wrap]
{\normalfont\fontseries{b}\selectfont\filright}
{\textcolor{carmine}{\thesubsection}}{.5em}{}
\titlespacing{\subsection}
{12pc}{1.5ex plus .1ex minus .2ex}{1pc}
%%%%%%%%%%%%%%%%%%%%%%%%%%%


\usepackage{multicol}

\usepackage{sagetex}


\usepackage{tikz}
\usetikzlibrary{mindmap,trees}
\usetikzlibrary{angles, quotes, arrows, arrows.meta}
\usetikzlibrary{shadows}
\usetikzlibrary{backgrounds}
\usetikzlibrary{calc}
 \usetikzlibrary{decorations.markings}
 
 
 
 %%%%% Grafica suma enteros
\newcommand{\sumaderecha}[3]{%
\begin{tikzpicture}[out=45,in=135,relative,>=stealth]
\draw[<->] (#1-2,0)--(#2+2,0);
\foreach \x in {\number\numexpr#1-1\relax,...,\number\numexpr#2+1\relax}  
\draw[shift={(\x,0)},color=black] (0pt,2pt) -- (0pt,-2pt) node[below] {\footnotesize $\x$};
\fill (#1,0) circle (2pt);
\fill (#2,0) circle (2pt);

\pgfmathsetmacro{\End}{#2-1} 
 \draw[#3,shorten >=2pt]
 \foreach \i in {#1,...,\End}{%
    (\i,0) to (\i+1,0)
} ; 
\node[color=black] at (#2,-0.75) {\small Final};
\node[color=black] at (#1,-0.75) {\small Inicio};
 \pgfmathsetmacro{\xtxt}{(#1+#2)/2}
\node at (\xtxt,0.5) {\small Moverse \number\numexpr#2-#1\relax\ unidades hacia la \emph{derecha}};
\end{tikzpicture}}

\newcommand{\sumaizquierda}[3]{%
\begin{tikzpicture}[out=135,in=45,>=stealth]
\draw[<->] (#2-2,0)--(#1+2,0);
\foreach \x in {\number\numexpr#2-1\relax,...,\number\numexpr#1+1\relax}
\draw[shift={(\x,0)},color=black] (0pt,2pt) -- (0pt,-2pt) node[below] {\footnotesize $\x$};
\fill (#1,0) circle (2pt);
\fill (#2,0) circle (2pt);

\pgfmathsetmacro{\End}{#2+1} 
 \draw[#3,shorten >=2pt]
 \foreach \i in {#1,...,\End}{%
    (\i,0) to  (\i-1,0)
} ; 
\node[color=black] at (#2,-0.75) {\small Final };
\node[color=black] at (#1,-0.75) {\small Inicio };
 \pgfmathsetmacro{\xtxt}{(#1+#2)/2}     
\node at (\xtxt,0.5) {\small Moverse \number\numexpr-#2+#1\relax\ unidades hacia la \emph{izquierda}};
\end{tikzpicture}} 


%%%%%%%% Fracciones

\newcount\segmentsleft
\tikzset{pics/.cd,
  circle fraction/.style args={#1/#2}{code={%
\segmentsleft=#1\relax
\pgfmathloop
\ifnum\segmentsleft<1\else
\ifnum\segmentsleft<#2 \edef\n{\the\segmentsleft}\else\def\n{#2}\fi
\begin{scope}[shift={(\pgfmathcounter,0)}]
\foreach \i [evaluate={\a=360/#2*(\i-1)+90;}] in {1,...,\n}
  \fill[fill=gray] (0,0) -- (\a:3/8) arc (\a:\a+360/#2:3/8) -- cycle;
\draw circle [radius=3/8];
\ifnum#2>1
  \foreach \i [evaluate={\a=360/#2*(\i-1);}] in {1,...,#2}
    \draw (0,0) -- (90+\a:3/8);
\fi
\end{scope}
\advance\segmentsleft by-#2
\repeatpgfmathloop
  }}
}

%%%%%%%%%%%%%%%%%%%%%%%%%%%%
\newcount\tikzfractiondenominator
\newcount\tikzfractionnumerator
\def\tikzfractionempty{}
\let\tikzfractionstyle=\tikzfractionempty
\newif\iftikzfractionfill
\tikzset{pics/.cd,
  fraction/.style={%
    code={%
      \tikzset{pics/fraction/.cd, #1}%
      \pgfmathparse{int(ceil(\tikzfractionnumerator/\tikzfractiondenominator))}%
      \let\tikzfractionshapetotal=\pgfmathresult
      \ifx\tikzfractionstyle\tikzfractionempty
      \else%
        \pgfmathloop
          \ifnum\tikzfractionnumerator<1
        \else
          \pgfmathsetmacro\tikzfractionproper{int(\tikzfractionnumerator?\tikzfractionnumerator:\tikzfractiondenominator)}%
          \foreach \tikzfractionsegmentnumber in {1,...,\tikzfractiondenominator}{%
            \ifnum\tikzfractionsegmentnumber>\tikzfractionproper\relax%
              \tikzfractionfillfalse%
            \else%
              \tikzfractionfilltrue%
            \fi%
            \let\tikzfractionshapenumber=\pgfmathcounter%
            \begin{scope}
              \tikzset{pics/fraction/\tikzfractionstyle/shape position/.try}%
              \tikzset{pics/fraction/\tikzfractionstyle/segment position/.try}%
              \tikzset{pics/fraction/\tikzfractionstyle/segment draw/.try}%
            \end{scope}
          }% 
          \advance\tikzfractionnumerator by-\tikzfractiondenominator%
        \repeatpgfmathloop%
      \fi%
    }
  },
  fraction/.cd,
    style/.store in=\tikzfractionstyle,
    numerator/.code=\pgfmathsetcount\tikzfractionnumerator{#1},
    denominator/.code=\pgfmathsetcount\tikzfractiondenominator{#1},
    fraction/.style args={#1/#2}{%
      /tikz/pics/fraction/.cd,
        numerator={#1}, denominator={#2}
    }
}
\tikzset{%
  /tikz/pics/fraction/triangles/.cd,
    shape position/.code={
      \pgfmathsetmacro\y{sqrt(\tikzfractiondenominator)}
      \tikzset{
        shift=(0:{(\tikzfractionshapenumber-1)*\y}),
        shift={(0,\y/4)},
      }
    },
    segment position/.code={
      \let\i=\tikzfractionsegmentnumber
      \pgfmathsetmacro\z{int(sqrt(\i-1))}
      \pgfmathsetmacro\q{\i-(\z)^2}
      \tikzset{
        shift={({sin(60) * (\q-\z) / 2}, {-\z*0.75 -mod(\q,2)*cos(60)/2})},
        rotate={mod(\q-1,2)*180}
      }
    },
    segment draw/.code={
      \iftikzfractionfill
        \tikzset{triangle fill/.style={blue!50!cyan!50}}
      \else
        \tikzset{triangle fill/.style={gray!20}}
      \fi
      \fill [triangle fill] (90:0.45) -- (210:0.45) -- (330:0.45) -- cycle;
    }
}
 
 
 %%%%%%%%% recta numeros mixtos
 % ref: WeCanLearnAnything at http://tex.stackexchange.com/questions/267921/macro-for-mixed-numbers-on-number-line-tikz (but I doubt this is the original source)
%\NewDocumentCommand\NL { s m m m m m }
%{\tikz[xscale=#2,yscale=#3]
% {
% \filldraw[orange] (0,0) rectangle (#6,0.2);% shaded portion of number line
% \draw
%  (0,0)--(#4,0)% lower part of x-axis
%  (0,0.2)--(#4,0.2);% higher part of x-axis
% \foreach \x in {0,...,#4}
%  \node [anchor=mid] at (\x,-0.5) {\x};% whole numbers underneath number line
%  \pgfmathparse{#4*#5}
% \foreach \x in {0,...,\pgfmathresult}% fractional tick marks and numbers above number line
%  {
%   \draw (\x/#5,-0.2)--(\x/#5,0.2);
%   \node[above] at (\x/#5,0.25) {$\frac{\x}{#5}$};
%   \IfBooleanF {#1}{
%     \pgfmathsetmacro\intbit{int(\x/#5)}
%     \pgfmathsetmacro\fracbit{int(\x-#5*\intbit)}
%     \ifnum\intbit=0\let\intbit\relax\fi
%     \ifnum\fracbit=0\else
%     \node [anchor=mid] at (\x/#5,-0.5) {$\intbit\frac{\fracbit}{#5}$};
%     \fi
%   }
%  }
% \fill[green,opacity=0.75] (#6,0.1) circle[x radius=0.2cm/#2,y radius=0.2cm/#3];% green dot
% }
%}


%% En el cuerpo
%\NL{4}{1.2}{3}{4}{7/4}
%\NL*{4}{1.2}{3}{4}{7/4}
 %{x-scale}{y-scale}{from 0 to 3}{denominator}{emphasized point}
 
 
 %%%%%%%%%%%%%%
 
 
\makeatletter
\newif\ifnl@mixednumbers
\tikzset{% http://tex.stackexchange.com/a/159856/ - Claudio Fiandrino
  number line/.code={
    \tikzset{
      /number line/.cd,%
      #1
    }
  },
  /number line/.cd,
  fraction/.store in=\nl@fraction,
  v scale/.store in=\nl@vscale,
  h scale/.store in=\nl@hscale,
  max/.store in=\nl@max,
  number to/.store in=\nl@numberto,
  mixed numbers/.is if=nl@mixednumbers,
  fill/.store in=\nl@fill,
  dot/.store in=\nl@dot,
  dot opacity/.store in=\nl@dotopacity,
  fraction=4,
  v scale=1.2,
  h scale=4,
  max=3,
  number to={7/4},
  mixed numbers=false,
  fill=orange,
  dot=green,
  dot opacity=.75,
}
\newcommand*\tnl{% modified from ref: WeCanLearnAnything at http://tex.stackexchange.com/questions/267921/macro-for-mixed-numbers-on-number-line-tikz (but I doubt this is the original source)
  \begin{scope}[xscale=\nl@hscale,yscale=\nl@vscale]
    \filldraw[\nl@fill] (0,0) rectangle (\nl@numberto,0.2);% shaded portion of number line
    \draw
    (0,0)--(\nl@max,0)% lower part of x-axis
    (0,0.2)--(\nl@max,0.2);% higher part of x-axis
    \foreach \x in {0,...,\nl@max}
      \node [anchor=mid] at (\x,-0.5) {\x};% whole numbers underneath number line
    \pgfmathparse{\nl@max*\nl@fraction}
    \foreach \x in {0,...,\pgfmathresult}% fractional tick marks and numbers above number line
    {
      \draw (\x/\nl@fraction,-0.2)--(\x/\nl@fraction,0.2);
      %\node[above] at (\x/\nl@fraction,0.25) {$\frac{\x}{\nl@fraction}$};
      \ifnl@mixednumbers
        \pgfmathsetmacro\intbit{int(\x/\nl@fraction)}
        \pgfmathsetmacro\fracbit{int(\x-\nl@fraction*\intbit)}
        \ifnum\intbit=0\let\intbit\relax\fi
        \ifnum\fracbit=0\else
          \node [anchor=mid] at (\x/\nl@fraction,-0.5) {$\intbit\frac{\fracbit}{\nl@fraction}$};
        \fi
      \fi
    }
    \fill[\nl@dot,opacity=\nl@dotopacity] (\nl@numberto,0.1) circle[x radius=0.2cm/\nl@hscale,y radius=0.2cm/\nl@vscale];% green dot
  \end{scope}}
\NewDocumentCommand \NumberLine { s O {} }{%
  \IfBooleanTF {#1}{%
    \tikz[number line={mixed numbers=false,#2}]\tnl;%
  }{%
    \tikz[number line={mixed numbers=true,#2}]\tnl;%
  }}
\makeatother

%%%%%%%%%%%%%%%%%%
 
 
 
 
 
 \begin{document}

%Se necesitan las imagenes enae.jpeg y mani.png
%\hspace*{-1cm}\includegraphics[height=1.6cm]{img/\string"enae\string".jpeg}\hspace*{15cm}  \includegraphics[height=1.6cm]{img/\string"mani\string".png}


%\begin{center}I. E. Escuela Nacional Auxiliares de Enfermeria\\
%                Matemáticas - S\'eptimo \\
%                Suma de n\'umeros enteros\\
%                Docente: Yamit Lopez Muñoz\\
%                E-mail: \href{mailto:yamitlopezm@gmail.com}{yamitlopezm@gmail.com}
%                
%                \end{center}
%                
%
%
%\hspace*{-1.8cm} { \small Lee atentamente \ClockFrametrue\ClockStyle=1\clock{12}{3} \textit{3 minutos.}} $\curvearrowright$  
%
%\begin{tcolorbox}
%INDICACIONES:
%
%\begin{itemize}
%\item Bla bla
%\item Bla bla
%\item Bla bla
%\item Bla bla
%\end{itemize}
%
%\end{tcolorbox}
%
%
%\section{Vivencia}
%


%\renewcommand{\theenumi}{\arabic{enumi}} %Números romanos en minúscula

\renewcommand{\labelenumi}{{\theenumi})}

\begin{sagesilent}
 
 
alumnas=[
  "AGUIRRE SALAZAR NIKOLLE".lower()   
,"ARCILA PUENTES VALERIA".lower()
,"BEDOYA AGUDELO NABRI SAMANTHA".lower()
,"BERMUDEZ BAÑOL VALERIA".lower()
,"BUSTOS OROZCO ANA ISABEL".lower() 
#,"CARDENAS QUINTERO ISABELA".lower() 
,"CHIVATA CARMONA SARA SOFIA".lower()
,"CORTES MONTOYA ISABELLA".lower() 
,"GARCIA MONSALVE NICOLE".lower() 
,"GARCIA PEREZ MANUELA".lower() 
,"GAVIRIA SOTO JULIANA".lower() 
,"GOMEZ RUIZ JUANITA".lower() 
,"GUTIERREZ TANGARIFE SALOME".lower() 
,"HERNANDEZ VERGARA SAMANTHA".lower() 
,"HERRERA SALINAS SALOME".lower() 
,"JARAMILLO PEREZ HELEN DAYANNA".lower()
,"JARAMILLO RAMIREZ SALOME".lower() 
,"LOPEZ GOMEZ MARIA VICTORIA".lower()
,"LOPEZ MARIN PAULA ANDREA".lower()
,"LOPEZ ORTEGON LUCIANA".lower() 
,"MARIN DIAZ ISABEL SOFIA".lower()
,"MEJIA CARDENAS LAUREN DAYANA".lower()
,"PARRA ROBLEDO ISABELA".lower() 
,"RAMIREZ ROJAS CAMILA".lower() 
,"RAMIREZ ROJAS CAROLINA".lower() 
,"RENDON VARGAS MARIA FERNANDA".lower()
,"RIVERA PATIÑO SALOME".lower() 
,"ROMAN MARIN ESTEFANY".lower() 
,"SALAZAR ARANGO MARIANA".lower() 
,"SALAZAR POSADA VALENTINA".lower() 
,"SANABRIA OSPINA SOFIA".lower() 
,"VALENCIA AGUIRRE MARIA JOSE".lower()
,"VALENCIA BURGOS VALERIA".lower() 
,"VALENCIA ORTEGA  ANA CAMILA".lower()
]

def signo(a):
   if a>=0:
      return str(a)
   else:
      aux= "(" + str(a) + ")"
      return aux   


problema =  r""
sol = r""

  
for i in range(0, len(alumnas)):
   problema += r"\begin{tcolorbox}[colframe=white, coltitle=black, colback=white, title= 7-1 %s) %s (Suma de fracciones)]"%(latex(i+1),   latex(alumnas[i].upper()))
   sol += r"\begin{tcolorbox}[colframe=white, coltitle=black, colback=white, title= 7-1 %s) %s (Suma de fracciones - Solución)]"%(latex(i+1),   latex(alumnas[i].upper()))
   problema += r"\begin{itemize}"
   sol += r"\begin{itemize}"
   problema += r"\item Halle el mínimo común múltiplo de los denominadores para realizar la suma indicada:"
   sol += r"\item Halle el mínimo común múltiplo de los denominadores para realizar la suma indicada:"
   problema += r"\begin{enumerate}"
   sol += r"\begin{enumerate}"
   problema += r"\begin{multicols}{2}"
   sol += r"\begin{multicols}{2}"
	   
   for j in range(0,80):
       b = choice(list(range(-20,0))+list(range(1,21)))
       a = choice(list(range(-20,0))+list(range(1,21)))
       d = choice(list(range(-20,0))+list(range(1,21)))
       c = choice(list(range(-20,0))+list(range(1,21)))
       m=lcm(b,d)
       r = m/b
       s = m/d
       problema += r"\item $\frac{%s}{%s}+\frac{%s}{%s}=$"%(a,b,c,d)
       sol += r"\item $\frac{%s}{%s}+\frac{%s}{%s}=\frac{%s(%s)+%s(%s)}{%s}=\frac{%s+%s}{%s}=\frac{%s}{%s}$"%(a,b,c,d,a,r,c,s,m,a*r, signo(c*s),m,a*r+c*s,m)
   problema += r"\end{multicols}"   
   sol += r"\end{multicols}"
   problema += r"\end{enumerate}"
   sol += r"\end{enumerate}"   
   problema += r"\end{itemize}"       
   sol += r"\end{itemize}"     
   problema += r"\end{tcolorbox}"
   sol += r"\end{tcolorbox}"
   

\end{sagesilent}

%\begin{enumerate}
%\item Escriba la fracción que representa cada una de las siguientes gráficas
%
%\begin{tikzpicture}
%\foreach \numerator/\denominator [count=\y] 
%in {1/1, 1/3, 2/4, 3/5, 8/8, 4/1, 10/3, 20/6, 30/7, 40/15}{
%\node at (-1/2,-\y) {$\frac{\numerator}{\denominator}$};
%\pic  at (0, -\y) {circle fraction={\numerator/\denominator}};
%}
%\end{tikzpicture}
%
%\tikz{ \node at (-0.5,0) {$\frac{\quad}{\quad}$}; \pic  at (0, 0) {circle fraction={1/2}};}
%
%\end{enumerate}


%\begin{tikzpicture}
%\foreach \numerator/\denominator [count=\y]  in {1/1, 2/4, 13/9}{
%\tikzset{shift=(270:\y*2)}
%\pic {fraction={style=triangles, fraction={\numerator/\denominator}}};
%\node at (-1,0)  {$\frac{\numerator}{\denominator}$};
%}
%\end{tikzpicture}
%
%
%\tikzset{%
%  /tikz/pics/fraction/petals/.cd,
%    shape position/.code={
%      \tikzset{
%        shift=(360/\tikzfractionshapetotal*\tikzfractionshapenumber:2)
%      }
%    },
%    segment position/.code={
%      \tikzset{
%        rotate=(360/\the\tikzfractiondenominator*\tikzfractionsegmentnumber)
%      }
%    },
%    segment draw/.code={
%      \iftikzfractionfill
%        \tikzset{petal/.style={bottom color=purple, top color=pink}}
%      \else
%        \tikzset{petal/.style={bottom color=yellow!50, top color=orange!50}}
%      \fi
%      \pgfmathparse{180/\tikzfractiondenominator}%
%      \let\r=\pgfmathresult
%      \path [petal] (0:0) [rounded corners=1ex] -- 
%        (-\r:0.5) -- (0:.75) -- (\r:0.5) -- cycle;
%    }
%}
%
%
%
%\begin{tikzpicture}
%\pic {fraction={style=petals, fraction={53/8}}};
%\node {$\frac{53}{8}$};
%\end{tikzpicture}
%
%
%\begin{tikzpicture}
%\pic {fraction={style=petals, fraction={11/5}}};
%\node {$\frac{\quad}{\quad}$};
%\end{tikzpicture}
%
%Dibuja y rotula una recta numérica del $0$ al $3$ con marcas en cada cuarto, enfatizando  los números mixtos, por ejemplo: \(\frac{7}{4}=1\frac{3}{4}\). Muestre su respuesta usando una longitud y un punto
%
%
%
%%\NumberLine
%
%%\NumberLine*
%
%\NumberLine[dot=green,dot opacity=.75,fill=orange,max= 3,fraction=2,mixed numbers=false, number to={4/2},h scale=5,v scale = 1.2]
%
%
%\NumberLine[dot=green,dot opacity=.75,fill=orange,max= 3,fraction=5,mixed numbers=false,number to={7/5},h scale=5,v scale = 1.2]
%
%\NumberLine[dot=green,dot opacity=.75,fill=orange,max= 2,fraction=10,mixed numbers=false,number to={1/10},h scale=10,v scale = 1.2]
%


%\tikzset{
%  number line={
%    fraction=5,
%    mixed numbers=,
%    number to={6/12},
%    h scale=8,
%    v scale=1.25,
%    max=2,
%  }
%}
%
%\begin{tikzpicture}
%  \tnl;
%  \begin{scope}[yshift=-30mm]
%    \tnl;
%  \end{scope}
%\end{tikzpicture}


%\begin{tikzpicture}
%    \filldraw[gray!50] (0,0) rectangle (2.5,.5) ; 
%    \draw[step=5mm] (0,0) grid (4.5,.5) ;
%    \foreach \x [evaluate=\x as \xpos using \x*1.5] in {0,...,3}
%      { \draw (\xpos,0)--++(0,-.2) node[below] {\x} ; }
%  \end{tikzpicture}
%  
%  \usetikzlibrary{svg.path}
%  
%  \begin{tikzpicture}
%  \draw svg "M108 28v40l-60 60-20-60z";
%  \end{tikzpicture}
%

%\begin{titlepage}
%\textbf{\Huge Me} 
%\end{titlepage}

% \part*{heading } 
% \chapter*{heading } 
% \section*{heading } 
% \subsection*{heading } 
% \subsubsection*{heading } 
% \paragraph*{heading } 
% \subparagraph*{heading }
 
 
%\textbf{¿Qué es una fracción?}
%
%
%
%Una fracción representa el número de partes que cogemos de una unidad que está dividida en partes iguales. Se representa por dos números separados por una línea de fracción.
%
%\begin{center}
%\begin{tikzpicture}
%\fill[gray!60] (0,1) rectangle (1,2);
%\foreach \x in {0,1}{
%\foreach \y in {0,1}
%\draw (\x, \y) rectangle (\x+1, \y+1);
%}
%\draw (3.5, 1)  node {\Huge $\frac{1}{4}$};
%\end{tikzpicture}
%\end{center}
%
%\textbf{Términos de una fracción}
%
%Los términos de una fracción son el numerador y el denominador. El numerador es el número de partes que tenemos y el denominador es el número de partes en que hemos dividido la unidad.
%
%Veamos algunos ejemplos: Tenemos diferentes figuras que representan la unidad, cada una de ellas la dividimos en partes iguales que determinan el denominador. La partes coloreadas representan el numerador.
%\begin{center}
%\begin{tikzpicture}[scale=0.8]
%\begin{scope}[shift={(1,3)}]
%\fill[gray!60] (0,2) rectangle (2, 3);
%\draw (0,0) grid[xstep=2 , ystep=1 ] ++(2,3);
%\draw (1, -1) node {\Huge $\frac{1}{3}$};
%\end{scope}
%\begin{scope}[shift={(5,3)}]
%\fill[gray!60] (0.5, 0)--(0.5, 1.5)--(0, 1.5)--(0,3)--(2, 3)--(2,0)--cycle; 
%\draw (0,0) grid[xstep=0.5 , ystep=1.5 ] ++(2,3);
%\draw (1, -1) node {\Huge $\frac{7}{8}$};
%\end{scope}
%\begin{scope}[shift={(9,3)}]
%\fill[gray!60] (0,3)-- (2,0)--(0,0) -- cycle;
%\draw (0,0) rectangle ++(2,3);
%\draw (0,3) -- (2,0); 
%\draw (1, -1) node {\Huge $\frac{1}{2}$};
%\end{scope}
%\begin{scope}[shift={(13,3)}]
%\fill[gray!60] (0,0)-- (0,3)--(2/4,3)--(2/4, 0) -- cycle;
%\draw (0,0) grid[xstep=2/4 , ystep=3 ] ++(2,3);
%\draw (1, -1) node {\Huge $\frac{1}{4}$};
%\end{scope}
%\begin{scope}[shift={(17,3)}]
%\fill[gray!60] (0,3/5)-- (0,3)--(2,3)--(2, 3/5) -- cycle;
%\draw (0,0) grid[xstep=2 , ystep=3/5 ] ++(2,3);
%\draw (1, -1) node {\Huge $\frac{4}{5}$};
%\end{scope}
%\end{tikzpicture}
%\end{center}
%
%
%\textbf{¿Cómo se leen las fracciones?}
%
%El numerador se lee con los números cardenales. $1$ – un, $2$ – dos, $3$ – tres, …, $10$ – diez, …, $24$ – veinticuatro…
%
%El denominador se lee con los números partitivos. $2$ – medios, $3$ – tercios, $4$ – cuartos, $5$ – quintos, $6$ – sextos, $7$ – séptimos, $8$ – octavos, $9$ – novenos, $10$ – décimos. A partir del $11$, el número se lee terminado en -avos: $11$ – onceavos, $12$ – doceavos, …
%
%\textbf{Tipos de fracciones}
%\begin{itemize}
%\item Fracción propia: Cuando el numerador es menor que el denominador.
%\item Fracción impropia: Cuando el numerador es mayor que el denominador.
%\item Fracción unitaria: Cuando el numerador es igual que el denominador.
%\end{itemize}



\sagestr{problema}
\sagestr{sol}





\end{document}

